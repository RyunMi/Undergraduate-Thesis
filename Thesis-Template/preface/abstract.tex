%----摘要----------------------------------------------------------------------------------------
\phantomsection\addcontentsline{toc}{chapter}{摘要}\tolerance=500 %将摘要放进目录
\begin{cnabstract}
本文介绍湘潭大学论文模板 的使用方法。本模板符合学校的本科论文格式要求。

摘要二字中间空两个汉字空格或4个半角空格,中文字体为宋体,西文字体为Times New Roman,三号,加粗,大纲级别1级,居中无缩进,段前0行,段后0行,单倍行距。另起一页。

摘要内容为中文字体为宋体,西文字体为Times New Roman,五号,大纲级别正文文本,两端对齐,首行缩进2字符,段前0.5行,段后0.5行,固定值20磅。

“关键词:”宋体小四号加粗,其后关键词中文字体为宋体,西文字体为Times New Roman,五号,大纲级别正文文本,两端对齐,首行缩进2字符,段前0行,段后0行,固定值20磅。每一关键词之间用全角分号隔开(;)最后一个关键词后不打标点符号。关键词前空一空白行。
\end{cnabstract}
~\par
\begin{cnkeywords}
关键词1;关键词2
\end{cnkeywords}
%----Abstract------------------------------------------------------------------------------------
\newpage
\phantomsection\addcontentsline{toc}{chapter}{Abstract}\tolerance=500 %将摘要放进目录
\begin{enabstract}
This article introduces how to use the thesis template of Xiangtan University. This template complies with the university's undergraduate dissertation formatting requirements.

There should be two Chinese character spaces or 4 half-width spaces in the middle of the word abstract. The Chinese font is Times New Roman, and the Western font is Times New Roman, size 3, bold, outline level 1, centered without indentation, 0 lines before a paragraph, paragraph After 0 lines, single line spacing. Start another page.

The content of the abstract is Song typeface in Chinese, Times New Roman in Western font, size 5, outline-level body text, justified, 2 characters indentation for the first line, 0.5 lines before a paragraph, 0.5 lines after a paragraph, and a fixed value of 20 points.

"Keywords:" Song Ti small four bold, followed by the Chinese font is Song Ti, Western font is Times New Roman, five, outline level body text, justified at both ends, the first line indented 2 characters, before the paragraph 0 lines, 0 lines after the paragraph, fixed value 20 points. Separate each keyword with a full-width semicolon (;) without a punctuation mark after the last keyword. There is a blank line before the keywords.
\end{enabstract}
~\par
\begin{enkeywords}
keyword1; keyword2
\end{enkeywords} 